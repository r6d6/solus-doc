% ----------------------------------------------------------
% Introdução (exemplo de capítulo sem numeração, mas presente no Sumário)
% ----------------------------------------------------------
\chapter{Introdução}
\addcontentsline{toc}{chapter}{Introdução}
% ----------------------------------------------------------

\section{Contextualização}

Na sociedade contemporânea, ao percebermos o problema por trás das fontes de energia elétrica comumente utilizadas, diversas fontes de energia menos degradantes ao meio ambiente vem sendo estudadas.

Em meio ao estudo de diferentes fontes de energia a energia solar vem mostrando seu valor, se sobresaindo quando analisadas questões como facilidade de instalação e manutenção, discrição e custo. Visto que, outras fontes de energia renovavéis são de díficil acesso.

Porém mesmo com as vantagens da utilização de energia solar, a instalação não é trivial, e precisa ser feita sobre uma prévia análise das condições climáticas da região, visando maior aproveitamento dos recursos investidos.

\section{Tema}

O mote acerca do desenvolvimento desse trabalho, gira entorno do auxílio da tomada de decisão através do desenvolvimento de um software de análise e visualização de dados.

\section{Objetivo}

Este desenvolvimento tem como objetivo introduzir a questão da análise de dados meteorológicos e tomada de decisão acerca da instalação de painéis solares e documentar através de metodologias de descrição de software, a arquitetura e desenvolvimento de uma aplicação que visa auxiliar na resolução deste problema.

\section{Delimitação do Problema}

Não existe hoje uma forma prática, de baixo custo e precisa de realizar a análise de dados ante a instalação de painéis solares, visando, com uma massa de dados dispostos de maneira simples e organizada, auxiliar a tomada de decisão na instalação destes painéis, procurando através de variáveis como posição, ângulo e outras, extrair o máximo de performance na captação de energia solar.

\section{Justificativa}

Existe um projeto de instalação de uma usina solar no IFSP, no campus localizado em Boituva, portanto, o tema do projeto foi escolhido, para que se possa através da análise prévia de dados, se encontrar a melhor configuração para os painéis fotovoltaicos, atingindo assim, uma maior captação de energia e aproveitamento de investimento.

\section{Método}

A metodologia de trabalho escolhida para este projeto, utiliza algumas convenções da metologia SCRUM, porém, pelo tamanho limitado da equipe, o projeto foi trabalhado sendo ditado pela metodologia KANBAN.
Para medirmos a qualidade do software, foi decidido optar pelo desenvolvimento da aplicação utilizando-se de técnicas de controle de qualidade orientadas ao teste.
