\chapter{Descrição Geral do Sistema}
\addcontentsline{toc}{chapter}{Descrição Geral do Sistema}

A arquitetura da aplicação desenvolvida pode ser resumida em dados sendo capturados através de sensores conectados a um microcontrolador arduino, serão captados dados de umidade do ar, temperatura e incidência de radiação solar.

Dados esses, que serão enviados através de requisições HTTP para uma API, serão armazenadas em banco de dados e então, será feita uma análise estatística dessas informações.

A interface do usuário final com a aplicação, será feita através de uma aplicação web, onde os dados analisados serão disponibilizados e o usuário fará consultas a essas informações.

\section{Principais Envolvidos e suas Características}

\subsection{Usuários do Sistema}

O sistema visa atender especialistas que precisam realizar tomadas de decisão.

Isso inclui também, clientes que, antes de realizar a instalação de painéis solares, precisam analisar se o investimento será compensado. E também, empresas de instalação de painéis solares, que gostariam de fazer uma análise de viabilidade mudando local, angulo e fazendo outras pesquisas acerca da instalação ou da manutenção de painéis fotovoltaicos.

\subsection{Desenvolvedores do Sistema}

Os envolvidos no desenvolvimento do projeto, são o orientador, Dr. Marcelo Polido, que ficará responsável pelos requisitos do sistema, ele irá coordenar o que será implementado e irá ditar as entregas incrementais. Também responsável por requisitos do projeto está o Professor Mario Pin, que será algo próximo de um Product Owner, ele será o primeiro cliente final da aplicação, irá utilizar o sistema para realizar análise de dados.

O planejamento e desenvolvimento do projeto, ficará por conta do aluno responsável pela defesa do mesmo, Angelo Silva.

O projeto é open source, ou seja, aberto para a comunidade no github, recebendo então, pequenas contribuições esporádicas de outros desenvolvedores ao longo do ciclo de vida do projeto.

\subsection{Tecnologias Empregadas}

A aplicação foi desenvolvida utilizando arduino para gerenciamento e captura dos dados utilizando requisições HTTP através das libraries do arduino. A conexão com a internet foi feita utilizando um arduino ethernet shield wifi, as informações são capturadas, é feita uma validação e formatação básica dos dados e então os mesmos são enviados para uma RESTFUL API construída com NODEJS.

O banco de dados utilizado foi o MYSQL, sendo servido a engine XtraDB, servido pela aplicação percona server.

\section{Regras de Negócio}

A maior parte das regras de negócio de sistema fica centralizada nos filtros, eles são quem valida os dados do sistema, definindo regras para a captura de dados.
A regra de negócio de filtragem diz que os dados de umidade e temperatura não podem se diferenciar por 50\% ou mais da média dos dados captados na ultima hora.
