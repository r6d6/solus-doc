% Apresentações em widescreen. Outros valores possíveis: 1610, 149, 54, 43 e 32.
% Por padrão, as apresentações são no formato 4:3 (sem o aspectratio).
\documentclass[aspectratio=169]{beamer}

\usetheme{Pittsburgh}
\usecolortheme{default}
\usefonttheme[onlymath]{serif}			% para fontes matemáticas
% Enconte mais temas e cores em http://www.hartwork.org/beamer-theme-matrix/
% Veja também http://deic.uab.es/~iblanes/beamer_gallery/index.html

% Customizações de Cores: fg significa cor do texto e bg é cor do fundo
\setbeamercolor{normal text}{fg=black}
\setbeamercolor{alerted text}{fg=red}
\setbeamercolor{author}{fg=black}
\setbeamercolor{institute}{fg=blue}
\setbeamercolor{date}{fg=black}
\setbeamercolor{frametitle}{fg=red}
\setbeamercolor{framesubtitle}{fg=brown}
\setbeamercolor{block title}{bg=blue, fg=white}		%Cor do título
\setbeamercolor{block body}{bg=gray, fg=darkgray}	%Cor do texto (bg= fundo; fg=texto)

% ---
% PACOTES
% ---
\usepackage[alf]{abntex2cite}		% Citações padrão ABNT
\usepackage[brazil]{babel}		% Idioma do documento
\usepackage{color}			% Controle das cores
\usepackage[T1]{fontenc}		% Selecao de codigos de fonte.
\usepackage{graphicx}			% Inclusão de gráficos
\usepackage[utf8]{inputenc}		% Codificacao do documento (conversão automática dos acentos)
\usepackage{txfonts}			% Fontes virtuais
% ---

% --- Informações do documento ---
\title{Apresentação da documentação do sistema Solus}
\author{Angelo Silva}
\institute{Instituto Federal de Educação, Ciência e Tecnologia de São Paulo Câmpus Boituva
	    \par
	    Curso de Análise e Desenvolvimento de sistemas}
\date{\today, v-0.0.1}
% ---

% ----------------- INÍCIO DO DOCUMENTO --------------------------------------
\begin{document}

\frame{\titlepage}

% ----------------- NOVO SLIDE --------------------------------
\begin{frame}{Sumário}
\tableofcontents
\end{frame}

% ----------------- NOVO SLIDE --------------------------------
\begin{frame}{Construção}
\section{Construção}

A documentação do projeto foi construída utilizando lateX e o framework abnteX.

\vspace{0.1in}

O projeto pode ser encontrado em: \textbf{github.com/r6d6/solus}

\end{frame}

\begin{frame}{Introdução}
\section{Introdução}

Na sociedade contemporânea, diversas preocupações quanto a captação de energia surgiram. Uma dessas preocupações é cada vez mais, buscar fontes renováveis de energia.

\vspace{0.1in}

Atualmente, a energia solar vem mostrando seus benefícios, sendo pelo custo, que é muito mais baixo do que pás eólicas e pela facilidade de instalação, que pode ser feita sem a necessidade de uma grande área reservada.

\vspace{0.1in}

Devido ao avanço da captação de energia solar, diversos desafios surgiram ao se estudar a melhor forma de se trabalhar com a energia captada.

\end{frame}

% --- O comando \allowframebreaks ---
% Se o conteúdo não se encaixa em um quadro, a opção allowframebreaks instrui
% beamer para quebrá-lo automaticamente entre dois ou mais quadros,
% mantendo o frametitle do primeiro quadro (dado como argumento) e acrescentando
% um número romano ou algo parecido na continuação.

% \begin{frame}[allowframebreaks]{Referências}
% \bibliography{abntex2-modelo-references}
% \end{frame}

% ----------------- FIM DO DOCUMENTO -----------------------------------------
\end{document}
