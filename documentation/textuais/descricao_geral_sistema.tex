\chapter{Descrição Geral do Sistema}
\addcontentsline{toc}{chapter}{Descrição Geral do Sistema}

O projeto visa, através da análise estatística de dados meteorológicos, auxiliar o estudo de viabilidade acerca da instalação de painéis fotovoltaicos.

Para isso, serão coletados dados através de sensores conectados a um microcontrolador arduino. Inicialmente, prevemos captar informações de umidade do ar, temperatura e incidência de radiação solar.

Dados esses, que serão enviados através de requisições HTTP para uma API, serão armazenadas em banco de dados e então, será feita uma análise estatística desses dados.

A interface do usuário final com a aplicação, será feita através de uma aplicação web, onde os dados analisados serão disponibilizados, onde o usuário fará consultas a essas informações.

\section{Descrição do Problema}

Durante o estudo de viabilidade acerca da instalação de painéis fotovoltaicos no IFSP, notou-se uma dificuldade na captura e análise dos dados para tomada de decisão, justificando assim, a necessidade da automatização desse processo, considerando também, a quantidade massiva de dados e as possíveis falhas de estimativa pelo cálculo humano.

Pelo ainda alto de custo de instalação de painéis solares, uma decisão errada na instalação de painéis solares poderia causar um dano financeiro imensurável.

O sistema afeta principalmente, a configuração dos painéis como ângulo, posição, local, entre outras variaveis que poderiam, através da análise de dados serem melhor decididas.

\section{Principais Envolvidos e suas Características}

\subsection{Usuários do Sistema}

O sistema visa atender especialistas que precisam realizar tomadas de decisão.

Clientes que, antes de realizar a instalação de painéis solares, precisam analisar se o investimento realizado será compensado. E também, empresas de instalação de painéis solares, que gostariam de fazer uma análise de viabilidade mudando local, angulo e fazendo outras pesquisas acerca da instalação ou da manutenção de painéis fotovoltaicos.

\subsection{Desenvolvedores do Sistema}

Os envolvidos no desenvolvimento do projeto, são o orientador, Dr. Marcelo Polido, que ficará responsável pelos requisitos do sistema, ele irá coordenar o que será implementado e irá ditar as entregas incrementais. Também responsável por requisitos do projeto está o Professor Mario Pin, que será o Product Owner, ele será o primeiro cliente final da aplicação, irá utilizar o sistema para realizar análise de dados.

O planejamento e desenvolvimento do projeto, ficará por conta do aluno responsável pela defesa do mesmo, Angelo Silva.

O projeto é open source, ou seja, aberto para a comunidade no github, recebendo então, pequenas constribuições esporádicas de outros desenvolvedores ao longo do cliclo de vida do projeto.

\subsection{Tecnologias Empregadas}

O projeto foi desenvolvido utilizando arduino para gerenciamento e captura dos dados utilizando requisições Http através das libraries do arduino. A conexão com a internet foi feita utilizando um arduino ethernet shield wifi, as informações são capturadas, é feita uma validação e formatação básica dos dados e então os mesmos são enviados para uma RESTFUL API construída com PHP.

A API do projeto foi desenvolvida utilizando microframework Lumen, utilizando banco de dados MYSQL e Percona server como SGBD, através de uma interface construída com framework front end bootstrap, javascript ES6 e sass, a interface foi feita utilizando conceitos de usabilidade e utilizando mobile first.

\section{Regras de Negócio}

%% Neste item devem ser descritas as regras de negócio relevantes para o sistema, como por exemplo, restrições de negócio, restrições de desempenho, tolerância à falhas, volume de informação a ser armazenada, estimativa de crescimento de volume, ferramentas de apoio, etc.
A definir
