\chapter{Descrição Geral do Sistema}
\addcontentsline{toc}{chapter}{Descrição Geral do Sistema}

% Este capítulo tem como objetivo descrever de forma geral o sistema, o escopo e as principais funções. A descrição geral do sistema deve abrange os itens a seguir.

O projeto visa, através da análise estatística de dados meteorológicos, auxiliar o estudo de viabilidade acerca da instalação de painéis fotovoltaicos.

Para isso, serão coletados dados através de sensores conectados a um microcontrolador arduino. Inicialmente, prevemos captar informações de umidade do ar, temperatura e incidência de radiação solar.

Dados esses, que serão enviados através de requisições HTTP para uma API, serão armazenadas em banco de dados e então, será feita uma análise estatística desses dados.

A interface do usuário final com a aplicação, será feita através de uma aplicação web, onde os dados analisados serão disponibilizados, onde o usuário fará consultas a essas informações.

\section{Descrição do Problema}

Durante o estudo de viabilidade acerca da instalação de painéis fotovoltaicos no IFSP, notou-se uma dificuldade na captura e análise dos dados para tomada de decisão, justificando assim, a necessidade da automatização desse processo, considerando também, a quantidade massiva de dados e as possíveis falhas de estimativa pelo cálculo humano.

Pelo ainda alto de custo de instalação de painéis solares, uma decisão errada na instalação de painéis solares poderia causar um dano financeiro imensurável.

O sistema afeta principalmente, a configuração dos painéis como ângulo, posição, local, entre outras variaveis que poderiam, através da análise de dados serem melhor decididas.

\section{Principais Envolvidos e suas Características}

\subsection{Usuários do Sistema}

Neste item deve ser descrito para qual tipo de empresa se destina o sistema, os tipos de usuários que utilizarão o sistema.

Estas informações são importantes para a definição de usabilidade G do sistema.

\subsection{Desenvolvedores do Sistema}

Neste item deve ser descrito os tipos de pessoas envolvidas em todo o desenvolvimento do sistema direta ou indiretamente.

Estas informações são importantes para a distribuição de responsabilidades e pontos-focais de desenvolvimento.

\subsection{Regras de Negócio}

Neste item devem ser descritas as regras de negócio relevantes para o sistema, como por exemplo, restrições de negócio, restrições de desempenho, tolerância à falhas, volume de informação a ser armazenada, estimativa de crescimento de volume, ferramentas de apoio, etc.
