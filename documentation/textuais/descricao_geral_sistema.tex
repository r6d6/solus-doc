\chapter{Descrição Geral do Sistema}
\addcontentsline{toc}{chapter}{Descrição Geral do Sistema}

O projeto visa, através da análise estatística de dados meteorológicos, auxiliar o estudo de viabilidade acerca da instalação de painéis fotovoltaicos.

Para isso, serão coletados dados através de sensores conectados a um microcontrolador arduino. Inicialmente, prevemos captar informações de umidade do ar, temperatura e incidência de radiação solar.

Dados esses, que serão enviados através de requisições HTTP para uma API, serão armazenadas em banco de dados e então, será feita uma análise estatística.

A interface do usuário final com a aplicação, será feita através de uma aplicação web, onde os dados analisados serão disponibilizados e o usuário fará consultas a essas informações.

\section{Descrição do Problema}

Durante o estudo de viabilidade sobre a instalação de painéis fotovoltaicos no IFSP, notou-se uma dificuldade na captura e análise dos dados para tomada de decisão, justificando assim, a necessidade da automatização desse processo, considerando também, a quantidade massiva de dados e as possíveis falhas de estimativa pelo cálculo humano.

Pelo alto de custo de instalação de painéis solares, uma decisão errada no estudo de viabilidade poderia causar um dano financeiro imensurável.

O sistema afeta principalmente, a configuração dos painéis como ângulo, posição, local, entre outras variaveis que podem afetar o desempenho energético.

\section{Principais Envolvidos e suas Características}

\subsection{Usuários do Sistema}

O sistema visa atender especialistas que precisam realizar tomadas de decisão.

Isso inclui também, clientes que, antes de realizar a instalação de painéis solares, precisam analisar se o investimento será compensado. E também, empresas de instalação de painéis solares, que gostariam de fazer uma análise de viabilidade mudando local, angulo e fazendo outras pesquisas acerca da instalação ou da manutenção de painéis fotovoltaicos.

\subsection{Desenvolvedores do Sistema}

Os envolvidos no desenvolvimento do projeto, são o orientador, Dr. Marcelo Polido, que ficará responsável pelos requisitos do sistema, ele irá coordenar o que será implementado e irá ditar as entregas incrementais. Também responsável por requisitos do projeto está o Professor Mario Pin, que será algo próximo de um Product Owner, ele será o primeiro cliente final da aplicação, irá utilizar o sistema para realizar análise de dados.

O planejamento e desenvolvimento do projeto, ficará por conta do aluno responsável pela defesa do mesmo, Angelo Silva.

O projeto é open source, ou seja, aberto para a comunidade no github, recebendo então, pequenas contribuições esporádicas de outros desenvolvedores ao longo do ciclo de vida do projeto.

\subsection{Tecnologias Empregadas}

A aplicação foi desenvolvida utilizando arduino para gerenciamento e captura dos dados utilizando requisições Http através das libraries do arduino. A conexão com a internet foi feita utilizando um arduino ethernet shield wifi, as informações são capturadas, é feita uma validação e formatação básica dos dados e então os mesmos são enviados para uma RESTFUL API construída com PHP.

A API do projeto foi desenvolvida utilizando microframework Lumen, utilizando banco de dados MYSQL e Percona server como SGBD, através de uma interface construída com framework front end bootstrap, javascript ES6 e sass, a interface foi construída seguindo conceitos de usabilidade.

\section{Regras de Negócio}

A maior parte das regras de negócio de sistema fica centralizada nos filtros, eles são quem valida os dados do sistema, definindo regras para a captura de dados
A regra de negócio de filtragem diz que os dados de umidade e temperatura não podem se diferenciar por 50\% ou mais da média dos dados captados na ultima hora.
