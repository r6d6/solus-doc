% ----------------------------------------------------------
% Introdução (exemplo de capítulo sem numeração, mas presente no Sumário)
% ----------------------------------------------------------
\chapter{Introdução}
\addcontentsline{toc}{chapter}{Introdução}
% ----------------------------------------------------------

Na sociedade contemporânea, diversas preocupações quanto a captação de energia surgiram. Uma dessas preocupações é cada vez mais, buscar fontes renováveis de energia.

Atualmente, a energia solar vem mostrando seus benefícios, sendo pelo custo, que é muito mais baixo do que pás eólicas e pela facilidade de instalação, que pode ser feita sem a necessidade de uma grande área reservada.

Devido ao avanço da captação de energia solar, diversos desafios surgiram ao se estudar a melhor forma de se trabalhar com a energia captada.

\section{Tema}

Captação de dados meteorológicos por sensores e microcontroladores arduino, para que se faça a análise estatística de dados.

\section{Objetivo do Projeto}

Conseguir a melhor obtenção e utilização de energia solar de painéis fotovoltaicos através de captação e análise prévia dos dados meteorológicos, dados esses que precisam ser disponibilizados da maneira mais fácil possível.

\section{Delimitação do Problema}

Não existe uma forma prática de realizar a análise dos dados antes da instalação de painéis solares, visto que, os dados captados por sensores, para que seja feita a análise, possui um fluxo muito alto de informações, assim, a necessidade de uma aplicação que faça a análise dessa quantidade massiva de dados, se faz evidente.

\section{Justificativa da Escolha do Tema}

Existe um projeto de instalação de uma usina solar no IFSP, no campus localizado em Boituva, portanto, o tema do projeto foi escolhido, para que se possa, no futuro, trabalhar a energia captada por painéis solares da melhor forma possível.

\section{Método de Trabalho}

A metodologia de trabalho escolhida para este projeto, utiliza a metodologia de desenvolvimento de software SCRUM com entregas incrementais, para a implementação do projeto, foi decidido a utilização de painéis microcontroladores arduino, enviando requisições HTTP para uma api, construída em PHP e utilizando banco de dados SQL.

\section{Organização do Trabalho}

Neste item deve-se descrever como o documento está organizado.

